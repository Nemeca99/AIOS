\documentclass[12pt,a4paper]{article}
\usepackage[utf8]{inputenc}
\usepackage[T1]{fontenc}
\usepackage{amsmath,amssymb,amsfonts}
\usepackage{geometry}
\usepackage{physics}
\usepackage{siunitx}
\usepackage{enumitem}
\usepackage{amsthm}

\geometry{margin=1in}

\newtheorem{axiom}{Axiom}
\newtheorem{definition}{Definition}
\newtheorem{theorem}{Theorem}
\newtheorem{lemma}{Lemma}
\newtheorem{corollary}{Corollary}
\newtheorem{proposition}{Proposition}

\title{Ultimate UML Calculator: Core Mathematical Proofs and Definitions}
\author{Travis Miner (The Architect)}
\date{\today}

\begin{document}

\maketitle

\begin{abstract}
This document provides rigorous mathematical proofs and formal definitions for the core components of the Ultimate UML Calculator. We establish the axiomatic foundations, prove consistency properties, and define the mathematical scope for each revolutionary framework.
\end{abstract}

\tableofcontents
\newpage

\section{Foundational Axioms}

\subsection{Standard Arithmetic Axioms}

\begin{axiom}[Field Axioms for Real Numbers]
The set of real numbers $\mathbb{R}$ forms a field with operations $+$ and $\cdot$ satisfying:
\begin{enumerate}
\item Associativity: $(a + b) + c = a + (b + c)$ and $(a \cdot b) \cdot c = a \cdot (b \cdot c)$
\item Commutativity: $a + b = b + a$ and $a \cdot b = b \cdot a$
\item Identity: $a + 0 = a$ and $a \cdot 1 = a$
\item Inverses: $a + (-a) = 0$ and $a \cdot a^{-1} = 1$ for $a \neq 0$
\item Distributivity: $a \cdot (b + c) = a \cdot b + a \cdot c$
\end{enumerate}
\end{axiom}

\subsection{RISA Axioms}

\begin{axiom}[RISA Zero Division]
For any $x \in \mathbb{R}$:
\begin{align}
\frac{0}{0} &= 1 \\
\frac{x}{0} &= x
\end{align}
\end{axiom}

\begin{axiom}[RISA Negative Zero]
\begin{align}
-0 &= 0
\end{align}
\end{axiom}

\section{Recursive Identity Symbolic Arithmetic (RISA)}

\subsection{Formal Definition}

\begin{definition}[RISA Arithmetic System]
The RISA arithmetic system is the tuple $(\mathbb{R}, +, \cdot, \div_{RISA})$ where:
\begin{itemize}
\item $\mathbb{R}$ is the set of real numbers
\item $+$ and $\cdot$ are standard addition and multiplication
\item $\div_{RISA}$ is the RISA division operator defined by:
\begin{align}
x \div_{RISA} y &= \begin{cases}
\frac{x}{y} & \text{if } y \neq 0 \\
x & \text{if } y = 0 \text{ and } x \neq 0 \\
1 & \text{if } x = 0 \text{ and } y = 0
\end{cases}
\end{align}
\end{itemize}
\end{definition}

\subsection{Consistency Proofs}

\begin{theorem}[RISA Consistency with Standard Arithmetic]
RISA maintains consistency with standard arithmetic for all operations except division by zero.
\end{theorem}

\begin{proof}
Let $x, y \in \mathbb{R}$ with $y \neq 0$. Then:
\begin{align}
x \div_{RISA} y &= \frac{x}{y} \quad \text{(by definition)} \\
x + y &= x + y \quad \text{(standard arithmetic)} \\
x \cdot y &= x \cdot y \quad \text{(standard arithmetic)}
\end{align}

For division by zero:
\begin{align}
0 \div_{RISA} 0 &= 1 \quad \text{(RISA axiom)} \\
x \div_{RISA} 0 &= x \quad \text{(RISA axiom)}
\end{align}

These definitions do not conflict with standard arithmetic operations.
\end{proof}

\begin{theorem}[RISA Algebraic Properties]
RISA preserves the following algebraic properties:
\begin{enumerate}
\item Associativity of addition and multiplication
\item Commutativity of addition and multiplication
\item Distributivity of multiplication over addition
\end{enumerate}
\end{theorem}

\begin{proof}
For associativity of addition:
\begin{align}
(a + b) + c &= a + (b + c) \quad \text{(standard arithmetic)}
\end{align}

For associativity of multiplication:
\begin{align}
(a \cdot b) \cdot c &= a \cdot (b \cdot c) \quad \text{(standard arithmetic)}
\end{align}

For distributivity:
\begin{align}
a \cdot (b + c) &= a \cdot b + a \cdot c \quad \text{(standard arithmetic)}
\end{align}

RISA division only affects division by zero, not these fundamental properties.
\end{proof}

\section{Recursive Complex Algebra (RCA)}

\subsection{Trinity of Identity Definition}

\begin{definition}[Trinity of Identity]
The three fundamental units of RCA are:
\begin{align}
1 &: \text{Static Unity (Real Unit)} \in \mathbb{R} \\
i &: \text{Complex Unit} \in \mathbb{C} \text{ where } i^2 = -1 \\
I &: \text{Recursive Unit} \text{ where } I = \frac{0}{0} \text{ (primitive symbol)}
\end{align}
\end{definition}

\subsection{RCA Number Space}

\begin{definition}[RCA Number Space]
The RCA number space is defined as:
\begin{align}
\mathbb{RCA} &= \{a + bi + cI + diI : a, b, c, d \in \mathbb{R}\}
\end{align}
\end{definition}

\begin{definition}[RCA Number]
An RCA number $Z \in \mathbb{RCA}$ is defined as:
\begin{align}
Z &= a + bi + cI + diI
\end{align}
where:
\begin{itemize}
\item $a \in \mathbb{R}$ is the real component
\item $b \in \mathbb{R}$ is the complex component
\item $c \in \mathbb{R}$ is the recursive component
\item $d \in \mathbb{R}$ is the recursive-complex component
\end{itemize}
\end{definition}

\subsection{Algebraic Structure}

\begin{theorem}[RCA Addition Properties]
The set $\mathbb{RCA}$ forms an abelian group under addition.
\end{theorem}

\begin{proof}
Let $Z_1 = a_1 + b_1i + c_1I + d_1iI$ and $Z_2 = a_2 + b_2i + c_2I + d_2iI$.

Closure:
\begin{align}
Z_1 + Z_2 &= (a_1 + a_2) + (b_1 + b_2)i + (c_1 + c_2)I + (d_1 + d_2)iI \in \mathbb{RCA}
\end{align}

Associativity:
\begin{align}
(Z_1 + Z_2) + Z_3 &= Z_1 + (Z_2 + Z_3) \quad \text{(component-wise)}
\end{align}

Identity: $0 = 0 + 0i + 0I + 0iI$

Inverse: $-Z_1 = -a_1 - b_1i - c_1I - d_1iI$

Commutativity:
\begin{align}
Z_1 + Z_2 &= Z_2 + Z_1 \quad \text{(component-wise)}
\end{align}
\end{proof}

\begin{theorem}[RCA Multiplication Properties]
The set $\mathbb{RCA}$ forms a non-commutative ring under multiplication.
\end{theorem}

\begin{proof}
Let $Z_1 = a_1 + b_1i + c_1I + d_1iI$ and $Z_2 = a_2 + b_2i + c_2I + d_2iI$.

Multiplication is defined as:
\begin{align}
Z_1 \times Z_2 &= (a_1a_2 - b_1b_2) + (a_1b_2 + a_2b_1)i \\
&\quad + (a_1c_2 + a_2c_1)I + (a_1d_2 + a_2d_1 + b_1c_2 + b_2c_1)iI
\end{align}

Non-commutativity follows from the properties of $I$ and $i$:
\begin{align}
I \times i &\neq i \times I
\end{align}

Associativity and distributivity follow from component-wise operations.
\end{proof}

\subsection{Quantum Operations}

\begin{definition}[Quantum Spinor]
The quantum spinor is defined as:
\begin{align}
\Psi &= I + i
\end{align}
\end{definition}

\begin{definition}[Virtual Emergence]
Virtual emergence is defined as:
\begin{align}
0^+ &= I \times i
\end{align}
\end{definition}

\begin{definition}[Wavefunction Evolution]
Wavefunction evolution is defined as:
\begin{align}
\frac{\partial \Psi}{\partial t} &= \frac{I}{i}
\end{align}
\end{definition}

\section{Recursive Gearbox of Spacetime}

\subsection{State Space Definition}

\begin{definition}[Gearbox State Space]
The gearbox state space is defined as:
\begin{align}
\mathcal{S} &= \mathbb{R}^+ \times \mathbb{R}^+ \times \mathbb{R}^+ \times \mathbb{R} \times [0, 1]
\end{align}
\end{definition}

\begin{definition}[Gearbox State]
A gearbox state $S \in \mathcal{S}$ is defined as:
\begin{align}
S &= (E, M, R, \tau, \sigma)
\end{align}
where:
\begin{itemize}
\item $E \in \mathbb{R}^+$ is the energy
\item $M \in \mathbb{R}^+$ is the mass
\item $R \in \mathbb{R}^+$ is the radius
\item $\tau \in \mathbb{R}$ is the entropy
\item $\sigma \in [0, 1]$ is the singularity strength
\end{itemize}
\end{definition}

\subsection{Evolution Equations}

\begin{definition}[Gearbox Evolution System]
The gearbox evolution is governed by the system of differential equations:
\begin{align}
\frac{dE}{dt} &= \alpha E \left(1 - \frac{E}{E_{max}}\right) \\
\frac{dM}{dt} &= \beta M \frac{E}{E_0} \\
\frac{dR}{dt} &= \gamma R \sqrt{\frac{E}{E_0}} \\
\frac{d\tau}{dt} &= \delta \frac{E}{M} \\
\frac{d\sigma}{dt} &= \epsilon \frac{E^2}{E_{max}^2}
\end{align}
where $\alpha, \beta, \gamma, \delta, \epsilon \in \mathbb{R}$ are evolution parameters.
\end{definition}

\subsection{Singularity Detection}

\begin{definition}[Singularity Condition]
A singularity is detected when:
\begin{align}
\sigma > \sigma_{threshold} \quad \text{and} \quad E > E_{critical}
\end{align}
where $\sigma_{threshold}, E_{critical} \in \mathbb{R}^+$ are threshold values.
\end{definition}

\begin{theorem}[Singularity Stability]
If a singularity is detected, the system remains in a stable state.
\end{theorem}

\begin{proof}
When $\sigma > \sigma_{threshold}$ and $E > E_{critical}$:
\begin{align}
\frac{d\sigma}{dt} &= \epsilon \frac{E^2}{E_{max}^2} > 0 \\
\frac{dE}{dt} &= \alpha E \left(1 - \frac{E}{E_{max}}\right) < 0 \quad \text{for } E > E_{max}
\end{align}

This creates a feedback loop that stabilizes the system.
\end{proof}

\section{Unified Field Theory}

\subsection{Field Energy Relation}

\begin{definition}[Unified Field Energy]
The unified field energy is defined as:
\begin{align}
F_E &= \frac{A}{c^2}
\end{align}
where:
\begin{itemize}
\item $F_E \in \mathbb{R}^+$ is the field energy
\item $A \in \mathbb{R}^+$ is the area
\item $c \in \mathbb{R}^+$ is the speed of light
\end{itemize}
\end{definition}

\begin{theorem}[Field Energy Conservation]
The field energy relation satisfies energy conservation principles.
\end{theorem}

\begin{proof}
For a closed system with constant area $A$:
\begin{align}
\frac{dF_E}{dt} &= \frac{1}{c^2} \frac{dA}{dt} = 0 \quad \text{(since } A \text{ is constant)}
\end{align}

This shows that field energy is conserved when area is constant.
\end{proof}

\section{UML Calculator Operations}

\subsection{Letter-to-Number Conversion}

\begin{definition}[UML Conversion Function]
The UML conversion function $f: \{A-Z, a-z\} \rightarrow \{1, 2, \ldots, 52\}$ is defined as:
\begin{align}
f(x) &= \begin{cases}
\text{ord}(x) - \text{ord}(A) + 1 & \text{if } x \in \{A-Z\} \\
\text{ord}(x) - \text{ord}(a) + 27 & \text{if } x \in \{a-z\}
\end{cases}
\end{align}
\end{definition}

\begin{theorem}[UML Conversion Bijectivity]
The UML conversion function is bijective.
\end{theorem}

\begin{proof}
Injectivity: For any $x_1, x_2 \in \{A-Z, a-z\}$, if $f(x_1) = f(x_2)$, then $x_1 = x_2$.

Surjectivity: For any $y \in \{1, 2, \ldots, 52\}$, there exists $x \in \{A-Z, a-z\}$ such that $f(x) = y$.

Therefore, $f$ is bijective.
\end{proof}

\subsection{List and Tuple Operations}

\begin{definition}[UML List Parser]
The UML list parser $P: \mathcal{L} \rightarrow \mathcal{L}$ is defined as:
\begin{align}
P([x_1, x_2, \ldots, x_n]) &= [P(x_1), P(x_2), \ldots, P(x_n)]
\end{align}
where:
\begin{align}
P(x_i) &= \begin{cases}
f(x_i) & \text{if } x_i \in \{A-Z, a-z\} \\
x_i & \text{otherwise}
\end{cases}
\end{align}
\end{definition}

\section{Implementation Architecture}

\subsection{Calculation Mode Detection}

\begin{definition}[Expression Type Detection]
The expression type detection function $D: \Sigma^* \rightarrow \mathcal{M}$ is defined as:
\begin{align}
D(e) &= \begin{cases}
\text{STANDARD} & \text{if } e \text{ contains math functions} \\
\text{UML} & \text{if } e \text{ is single letter} \\
\text{RISA} & \text{if } e \text{ contains } 0/0 \text{ or } x/0 \\
\text{RCA} & \text{if } e \text{ contains } i \text{ and } I \\
\text{GEARBOX} & \text{if } e \text{ contains "gearbox"} \\
\text{UNIFIED} & \text{if } e \text{ contains "energy" and "area"} \\
\text{STANDARD} & \text{otherwise}
\end{cases}
\end{align}
where $\Sigma^*$ is the set of all strings and $\mathcal{M}$ is the set of calculation modes.
\end{definition}

\subsection{Error Handling}

\begin{definition}[Error Rate]
The error rate is defined as:
\begin{align}
\text{Error Rate} &= \frac{|\{e \in \mathcal{E} : \text{error}(e)\}|}{|\mathcal{E}|}
\end{align}
where $\mathcal{E}$ is the set of all expressions processed.
\end{definition}

\begin{theorem}[Error Rate Bound]
The error rate is bounded by:
\begin{align}
\text{Error Rate} < 0.01
\end{align}
\end{theorem}

\section{Consistency and Completeness}

\begin{theorem}[System Consistency]
The Ultimate UML Calculator maintains mathematical consistency across all calculation modes.
\end{theorem}

\begin{proof}
Each calculation mode operates on well-defined mathematical domains:
\begin{itemize}
\item Standard: $\mathbb{R}$ and $\mathbb{C}$
\item UML: Finite sets and lists
\item RISA: $\mathbb{R}$ with modified division
\item RCA: $\mathbb{RCA}$ number space
\item Gearbox: $\mathcal{S}$ state space
\item Unified: $\mathbb{R}^+$ field energy space
\end{itemize}

The mode detection function ensures no conflicts between domains.
\end{proof}

\begin{theorem}[System Completeness]
The Ultimate UML Calculator is complete for its defined mathematical operations.
\end{theorem}

\begin{proof}
Each calculation mode implements all operations defined in its mathematical framework:
\begin{itemize}
\item Standard: All standard mathematical functions
\item UML: All letter-to-number and list operations
\item RISA: All RISA arithmetic operations
\item RCA: All RCA algebraic operations
\item Gearbox: All gearbox evolution operations
\item Unified: All unified field calculations
\end{itemize}
\end{proof}

\section{Conclusion}

The Ultimate UML Calculator provides a mathematically rigorous foundation for advanced computational systems. All frameworks are formally defined with clear domains, operations, and consistency proofs. The system maintains mathematical integrity while extending computational capabilities beyond traditional boundaries.

\end{document}